\newpage
\section{Grobner Bases}

We have seen that working with monomials ideals is easier than working with normal ideals. We want to somehow degenerate ideals to monomial ideals, which correspondingly degenerates schemes to unions of fuzzy coordinate hyperspaces. It turns out that degenerating an ideal to a corresponding monomial ideal preserves many properties, and either strictly improves/worsens some properties.

There are several examples of such properties that are excluded. We study the method of degeneration.

\subsection{Monomial Orders}

We first must define the notion of an ordering on the set of monomials. We identify monomials with elements of $\N^n$.

Recall our partial order of $\N^n$ given by $(a_1 , \cdots a_n) \leq (b_1 , \cdots b_n)$ if and only of $a_i \leq b_i $ for all $1 \leq i \leq n$.  We want to extend to a total order (ie one can compare any two elements).

\begin{definition}
A monomial order is a total order $<$ on $\N^n$ such that
\begin{enumerate}
    \item if $\alpha < \beta$ then $\alpha +\gamma < \beta + \gamma$ for all $\gamma \in \N^n$,
    \item $\forall \alpha \in \N^n, \alpha \geq (0,\cdots,0)$.
\end{enumerate}
\end{definition}

\begin{example}
There are three important examples.
\begin{itemize}
    \item lexicographic order (lex): $\alpha < \beta$ if first non-zero entry of $(\beta - \alpha)$ is positive
    \item graded lex order (grlex): $\alpha < \beta$ if either ($|\alpha| < |\beta|$) or ($|\alpha | = |\beta|$ and $\alpha <_{lex} \beta$)
    \item graded reverse lex (grevlex): $\alpha < \beta$ if either ($|\alpha| < |\beta|$) or ($|\alpha | = |\beta|$ and the last non-zero entry of $(\beta - \alpha)$ is negative)
\end{itemize}
\end{example}

\begin{example}
Let $x>y>z$. Consider $\set{x^2, xz^2, y^3}$. Then
\begin{itemize}
    \item[(lex):] $x^2 > xz^2 > y^3$
    \item[(grlex):] $xz^2 > y^3 > x^2$
    \item[(grevlex):] $ y^3 > xz^2 > x^2$
\end{itemize}
Now consider all monomials of degree $2$, $\set{x^2, y^2, z^2, xy,xz,yz}$. Then
\begin{itemize}
    \item[(grlex):] $x^2 > xy > xz > y^2 > yz > z^2$
    \item[(grevlex):] $ x^2 > xy > y^2 > xz > yz > z^2 $
\end{itemize}
\end{example}

\begin{proposition}
Let $<$ be a monomial order. Then:
\begin{enumerate}
    \item if $u,v$ are monomial with $u | v$ then $u \leq v$
    \item if $u_1 \geq u_2 \geq \cdots$ are monomials, then $\exists N$ such that $u_n = u_N$ for all $n \geq N$ 
\end{enumerate}
\end{proposition}

\subsection{Grobner Bases}

We move to Grobner Bases.

\begin{definition}
Let $S = K[ \underline{x}]$ and fix a monomial order $<$. For $0 \neq f \in S$. Define $in(f) = in_< (f)$ to be the $<$-greatest monomial of $supp(f)$.

We call $in(f) $ the initial/leading monomial, and the coefficient of $in(f)$ the initial/leading coefficient.

If $I$ is an ideal we set
\begin{align*}
    in(I) = ( \set{in (f) \mid f \in I}).
\end{align*}
\end{definition}

\begin{example}
Let $S = K[ x,y,z,w]$ with $x>y>z>w$ grevlex order. Let $I = (xy - zw, xz - y^2)$. Then
\begin{align*}
    in(I) = \set{xy, y^2, x^2z, \cdots}
\end{align*}
as $y(xy-zw) + x(xz - y^2)$.
\end{example}

We want a way to be able to handle these initial ideals easily.

By Dickson, we know that $in(I)$ is generated by a finite list of monomials. Note that each element generating $in(I)$ must be the initial monomials of some elements of $I$.

It follows that there must be some polynomials $g_1, \cdots, g_k \in I$ such that $in(I)$ is generated by $in(g_i)$.

\begin{definition}
A Grobner basis of $I$ is a set $G = \set{g_1, \cdots, g_n}$ such that
\begin{align*}
    in(I) = ( in(g_1) , \cdots in(g_n) ).
\end{align*}
\end{definition}

We can assume that $G $ generates $I$ as well, since $in(G)$ generates $in(I)$, and so we can add polynomials to $G$ until we get that $G$ generates $I$. Since adding polynomials doesn't affect that $in(G)$ generates $in(I)$ then this is fine.

\subsection{Several Hard Theorems}

\begin{theorem}
Let $I \subset S$ be an ideal. Then the residue classes of monomials in $S \setminus in(I)$ are a basis for $S / I$ as a $K$-vector space.
\end{theorem}

\begin{corollary}
If $I$ is a homogeneous ideal, then
\begin{align*}
    S/I = \bigoplus_{k=0}^\infty S_k / I_k
\end{align*} is graded and
\begin{align*}
    \dim( S/I)_k = \dim (S/ in(I) )_k.
\end{align*}
\end{corollary}

Recall the Hilbert series of $S$ is
\begin{align*}
    H(S ; t) = \sum_{i \in \N} (\dim S_i) t^i.
\end{align*}

\begin{theorem}
An ideal $I$ has only finitely many initial ideals (as in varying over infinite monomial orders gives only finite initial ideals). 
\end{theorem}

\begin{theorem}[Hilbert Basis Theorem]
Every ideal is finitely generated.

More precisely, if $g_1, \cdots, g_m$is a Grobner Basis of $I$, then $(g_1, \cdots, g_m) = I$.
\end{theorem}

\begin{corollary}
Let $I_1 \subset I_2 \subset \cdots$ be an ascending chain of ideals in $S$. Then $\exists N$ such that $I_k = I_N$ for all $k \geq N$.
\end{corollary}

\begin{proof}
$I_1 \subset I_2 \subset \cdots \implies in(I_1) \subset in(I_2) \subset \cdots$. We know that monomial ideals stabilize by prop \ref{monoStab}. By homework then we have that if $A \subset B$ and $in(A) = in(B)$ then $A=B$, and we are done.
\end{proof}

\subsection{Division Algorithm}

We have an algorithm for dividing polynomials. It is a bit hard to write down, so we hope that the reader knows it.

\begin{theorem}
Let $f \in S$ and $g_1, \cdots , g_m \in S$ be non-zero. Then the division algorithm produces polynomials $q_1, \cdots, q_m$ and $r$ such that
\begin{enumerate}
    \item $f = q_1 g_1 + \cdots + q_m g_m + r$
    \item no elements of $supp(r)$ is in $( in (g_1 ) , \cdots, in(g_m) )$
    \item $in ( q_i g_i) \leq in(f)$
\end{enumerate}
\end{theorem}

\begin{theorem}
Fix monomial order. Suppose $g_1, \cdots, g_m$ are a Grobner basis for $I$. Then every $f \in S$ has a unique remainder on division by $g_1 , \cdots, g_m$ (up to reordering $g_i$'s).
\end{theorem}

\begin{corollary}
Let $g_1 , \cdots , g_m $ be a Grobner basis for $I$ and let $f \in S$. Then $f \in I $ if and if $f$ has remainder $0$ upon division by $g_1, \cdots , g_m$.
\end{corollary}

\subsection{Buchberger's Algorithm}

\begin{definition}
Fix a monomial order. Then define
\begin{align*}
    S(f,g) = \frac{lcm ( in (f) , in(g) )}{c \cdot in (f)} f - \frac{lcm ( in (f) , in(g) )}{d \cdot in (g)} g 
\end{align*}
where $c,d$ are the leading coefficients of $f,g$ respectively.
\end{definition}

Algorithm: consider $I = (f_1, \cdots f_k) = (F)$. Then
\begin{enumerate}
    \item for each pair $f_i , f_j$ compute $S(f_i , f_j)$ and then divide it by $F$ to get $r$
    \item if all remainders are $0$, stop and declare this a Grobner basis, otherwise put $r$ into $F$ and repeat the algorithm
\end{enumerate}

This terminates in finite time.

\begin{theorem}
Fix a monomial order and let $I = (g_1, \cdots , g_m) $ with all $g_i \neq 0$. Then TFAE
\begin{enumerate}
    \item $g_1, \cdots , g_m $ is a Grobner basis for $I$
    \item each $S(g_i, g_j) $ reduces to $0 \mod  g_1, \cdots , g_m$
\end{enumerate}
\end{theorem}

\begin{lemma}
Fix a monomial order. Suppose that $\gcd (in(f) , in(g) ) = 1$. Then $S(f,g) $ reduces to $0 \mod f,g$.
\end{lemma}

\subsection{Reduced Grobner Bases}

\begin{definition}
Let $G = \set{g_1, \cdots , g_m}$ be a Grobner basis for $I$. Then we say that $G$ is reduced if
\begin{enumerate}
    \item all leading coefficients are $1$
    \item $\forall i \neq j $, no $u \in supp (g_i)$ is divisible by $in (g_j)$
\end{enumerate}
\end{definition}

\begin{theorem}
Reduced Grobner bases exist and are unique (up to choosing term order).
\end{theorem}