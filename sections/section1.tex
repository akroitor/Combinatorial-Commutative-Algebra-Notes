% !TEX root = ../CO739.tex
\newpage
\section{Ring Theory}

We first discuss rings.

\subsection{Algebraic Closure}

\begin{definition}[Algebraically Closed]
A field $K$ is called algebraically closed if each non-constant polynomial $f \in K[x]$ has a root.
\end{definition}

\begin{theorem}
Each field $K $ has an algebraic closure $\overline{K}$, the smallest algebraically closed field containing $K$.
\end{theorem}

\subsection{Polynomial Rings}

Let $K [ x_1 , \cdots , x_ n ] $ be a polynomial ring. We write $K [ x_1 , \cdots , x_ n ] = \vcentcolon K[ \underline{x} ]$ with
\begin{align*}
    \underline{x}^{\underline{a}} &= x_1^{a_1} \cdots x_n^{a_n}\\
    f &= \sum_{\underline{a} \in \N^n} c_{\underline{a}} \underline{x}^{\underline{a}}
\end{align*}
with only finitely many non-zero $c_{\underline{a}}$.

\begin{definition}
\begin{align*}
    \deg(f) = \max \set{ \abs{\underline{a}} \mid c_{\underline{a}} \neq 0}\\
    supp(f) = \set{ \underline{a} \mid c_{\underline{a}} \neq 0}.
\end{align*}
\end{definition}

Note that from now on we will be dropping underlines.

\begin{remark}
$K[x]$ is a graded ring.
\end{remark}

Note that in order to find the degree of a polynomial it doesn't suffice to just examine powers in it, one must write the polynomial in canonical form.

\begin{example}
\begin{align*}
    \deg \br{ (x+1)^2 -x^2 } = \deg ( 2x+1 ) = 1.
\end{align*}
\end{example}

\begin{definition}
A polynomial $f$ is homogeneous of degree $i$ if its only support is in degree $i$.
\end{definition}

\subsection{Homogeneous Polynomials}

Let $S = K[x]$ and $S_i \subset S$ be the subset of degree $i $ homogeneous polynomials (a finite dimensional vector space spanned by monomials of degree $i$). Then
\begin{align*}
    S = \bigoplus_{i=0}^\infty S_i
\end{align*}

\begin{remark}
$\dim S_i = \binom{i+n-1}{n-1}$.
\end{remark}

\begin{proof}
Stars and bars argument.
\end{proof}

\begin{definition}
The Hilbert series of $S$ is
\begin{align*}
    H(S ; t) = \sum_{i \in \N} (\dim S_i) t^i.
\end{align*}
\end{definition}

\begin{remark}
The Hilbert series of a polynomial ring is
\begin{align*}
    H( K[x] ; t) = \sum \binom{i+n-1}{n-1} t^i = \frac{1}{(1-t)^n}
\end{align*}
where the last equality is from the negative binomial theorem.
\end{remark}

\section{Ideals}

An ideal is the ring analogue of a normal subgroup.

\begin{definition}[Ideal]
An ideal is a non-empty subset $I \subseteq K[x]$ such that
\begin{enumerate}
    \item $f,g \in I \implies f+ g \in I$,
    \item $f \in I, g \in K[x] \implies gf \in I$.
\end{enumerate}
Note that if in the second condition $g \in K$ instead, $I$ is a vector subspace instead of an ideal.
\end{definition}

\subsection{Generating Ideals}

\begin{definition}[Ideal Generated by a Set]
For $F \subset K[x]$ we define $(F) $ (or $\langle F \rangle$) to be the smallest ideal containing $F$ and call it the ideal generated by $F$.
\end{definition}

\begin{proposition}
For $F \subset K [ x] $ , let $I_F$ be the set of all finite $K[ \underline{x} ]$-linear combinations of elements of $F$, that is to say all finite sums of the form $\sum g_i f_i$ with $g_i \in K[x], f_i \in F$. Then $I_F = (F)$.
\end{proposition}
\begin{proof}
We check in order
\begin{enumerate}
    \item $I_F$ satisfies conditions $1$ and $2$ of the definition of an ideal and is thus an ideal.
    \item $I_F$ contains $F$.
    \item Any ideal containing $F$ must contain $I_F$.
\end{enumerate}
\end{proof}

A natural question arises: given $f, g_1 ,\cdots , g_n$, how can we tell whether or not $f \in (g_1, \cdots, g_n)$? This is a non-trivial question.

\begin{example}
$K[\underline{x}] = \Q[x,y,z] , I = ( xy-z, yz-x, xz-y), f = z^3 - z$. It is not obvious yet true that $f \in I$.
\end{example}

\begin{definition}
An ideal is homogeneous (or graded) if its generated by homogeneous polynomials (of possibly different degrees).
\end{definition}

\begin{example}
$(x^2, x^3 + y^3, xy, x)$ is a homogeneous ideal.
\end{example}

\begin{proposition}
Let $I$ be an ideal of $S[\underline{x}]$. TFAE:
\begin{enumerate}
    \item $I$ is homogeneous.
    \item $f \in I \implies$ all homogeneous components of $f$ are in $I$.
    \item $I = \bigoplus_{j=0}^\infty I_j \,; \, I_j = I\cap S_j$ (recall $S_j$ is the set of homogeneous degree $j$ polynomials).
\end{enumerate}
\end{proposition}

\begin{proof}\phantom{.}
\begin{enumerate}
    \item[$1 \to 2$]: Define $G$ as the set of homogeneous generators, $f = \sum_i h_i g_i \in I, h_i \in S, g_i \in G$. $\deg (g_i) = d$, then $f_j = \sum_i h_i' g_i$ st $h_i'$ is part of $h_i$ that's homo of degree $j - d_i$. Then $f_j \in I$.
    \item[$2 \to 3$]: $f \in I, f_j \in I$, so $f_j \in I\cap S_j$, so $I = \sum_{j=0}^\infty I_j = \bigoplus_j I_j$
    \item[$3 \to 2$]: By definition.
    \item[$2 \to 1$]: Let $G$ be set of generators for $I$. Take the homo components of $G$. Then $F = \set{ g_j \mid g \in G, j \in \N}$.
\end{enumerate}
\end{proof}

\section{Quotient Rings}
Also known as residue class rings.

\begin{definition}
Let $I \subset S = K[\underline{x}]$ be an ideal. The residue class of $f \mod I$ is the set
\begin{align*}
    f + I = \set{f + i \mid i \in I}.
\end{align*}
The quotient ring $S/I $ is the st of all residue classes.
\end{definition}

\begin{remark}
$S/I$ is a ring with
\begin{align*}
    (f+ I) + (g+ I) &= (f+g) + I\\
    (f+ I) \cdot (g+ I) &= (fg) + I\\
\end{align*}
\end{remark}

\begin{proposition}
$f+ I = g+ I \iff (f-g) \in I$.
\end{proposition}

\begin{proof}
\phantom{.}
\begin{enumerate}
    \item[($\Leftarrow$)]: $f-g \in I \therefore f = g + i \, ; \, i + I = 0 + I \therefore f + I = (g+ i) + I = g+ I $.
    \item[($\Rightarrow$)]: $f + i \in g + I \therefore f+ i = g+j \therefore f-g = j-i \in I$.
\end{enumerate}
\end{proof}


\subsection{Ideal Operations}
We have the following operations between ideals (this list is non-exhaustive.
\begin{definition}
\phantom{.}
\begin{enumerate}
    \item[Sum: ] $I + J = \set{f + g \mid f \in I , g \in J}$,
    \item[Intersection: ] $I \cap J$,
    \item[Product: ] $IJ = ( \set{f g \mid f \in I , g \in J} )$,
    \item[Colon: ] $I : J = \set{f \in S \mid fj \in I \, \forall j \in J} = \set{f \in S \mid f J \subset I}$
\end{enumerate}
Note that the colon ideal of two ideals is also known as the ideal quotient of two ideals.
\end{definition}

\begin{proposition}
$IJ \subseteq I \cap J$
\end{proposition}

\begin{proof}
$f \in I, g \in J \implies fg \in I, fg \in J $ by the definition of an ideal. Thus $\set{ fg } \subset I \cap J \implies ( \set{ fg } ) \subset I \cap J$
\end{proof}

\begin{example}
Let $S = \Q[x] ; I = (x^3 + 6x^2 + 12x +8) = ( (x+2)^3 ) ; J = (x^2 + x - 2) = ( (x+2) ( x-1) )$. Then
\begin{align*}
    I+J &= (x+2) &(\gcd)\\
    I \cap J &= \br{ (x+2)^3 (x-1)  } &(\text{lcm})\\
    IJ &= \br{ (x+2)^4 (x-1)} &(\text{mult})\\
    I:J &= \br{ (x+2)^2} &\text{st.} (x+2)^2 \cdot (x+2) (x-1)  \sim (x+2)^3
\end{align*}
\end{example}

\begin{definition}
The radical of an ideal $I$
\begin{align*}
    \sqrt{I} = \set{ f \in S \mid f^k \in I , \text{ for some } k \geq 1}.
\end{align*}
\end{definition}

\begin{example}
$\sqrt{I} = (x + 2)$.
\end{example}
\begin{example}
$\sqrt{J} = \br{( x+2) (x -1)}$.
\end{example}

\begin{proposition}
$\sqrt{I}$ is an ideal.
\end{proposition}

\begin{proof}[sketch]
$(fp)^k = f^k p^k \in I$. $(f+g)^{k+l} = \sum_{k=0}^{k+i} c_i f^i g^{k+l -i} \in I $.
\end{proof}

\begin{definition}
$I$ is a radical ideal if $I = \sqrt{I}$.
\end{definition}

\begin{proposition}
For all ideals $I$, we have that $\sqrt{I} = \sqrt{\sqrt{I}}$.
\end{proposition}

\begin{proof}[sketch]
One side is obvious, then $(f^k)^l \in I$.
\end{proof}

% Recall that the goal of this course was to prove Dirichlet's Unit Theorem:

% \unit	% Dirichlet's Unit Theorem

% \pf L.T.R. \qed


% \begin{ex}
% If $K=\Q$, then $r=1$ and $s=0$ so that $r+s-1=0$. Therefore, $\O_\Q^\times=\Z^\times=\{\pm1\}$. Of course, this is the most trivial possible example. \xqed
% \end{ex}


% To learn even more Mathematics, read \cite{neu}. 